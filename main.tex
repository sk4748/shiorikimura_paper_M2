\documentclass[a4paper]{jsarticle}
\usepackage{iapaper}
\usepackage[dvipdfmx]{graphicx}


\begin{document}
% 修士論文の場合は \degreethesis を使わず,下記を使う.
% なお博士の場合は \doctorthesisにすること
 \masterthesis
% 卒業論文の場合は下記を使う
% \degreethesis


\title{論 文 題 目}
\date{平成28年度}
\advisor{渡邉英徳} % 指導教員名を入れる
\IDnumber{15893507}
\Mauthor{木村汐里}% 自分の名前を入れる.修士の場合は \Mauthor{氏名} に変更する.
\submissiondate{平成29年1月25日}
\maketitle


\pagenumbering{roman} % 要旨はRoman書体で表示
\setcounter{page}{1} % 1から振り直す
\jasummary{論文タイトル}{副題(必要であれば記入)}
ここには日本語文章のみで論文要旨を記述してください。
\summary{Title}{Subtitle}
The author shall write a thesis in English text. 150 words.

\makemokuji


\newpage

\pagenumbering{arabic}  % 論文本体はArabicで表示
\setcounter{page}{1} % ページ番号を1から振り直す
\section{序論}
本研究の概要, 背景と研究目的について述べ,本論文の構成について説明する.
\subsection{本研究の概要}
本研究では,集落組織が弱体化した過疎地域における「対話と交流の“場”」を形成する手法について検討する。そのために,新潟県魚沼市の過疎地域である横根地区に実際に滞在し,住民とともに“場”の形成を実践する.
まず,先行研究や先行事例を整理し、限界集落における地域活性化の活動における「対話と交流の“場”」の位置付けを明らかにする. さらにワークショップ手法の先行事例について検討し,”場”における話題を促進するコンテンツと,世代・性別ごとの参加プロセスに注視することの重要性を導き出す.
この結論に基づき,過疎地域における「対話と交流の“場”」のプログラムデザインを設計する.実際の集落においてこのプログラムを用いて“場”の形成を実践を通して比較検証した結果,地域の多様な世代が参加し,活発な意見が交換される場が形成された.さらに,活動の内容をSNSで地域外に発信し,広く共有することによって,地域内外が相互作用を生む場も形成された.
このことから,創出したプログラムデザイン手法は妥当であり、過疎地域における対話と交流の“場”が形成を促されたといえる.これは,専門家でなくとも利用可能であり今後,国内に増えていくと予想される限界集落における諸問題を解決するための,一つのモデルとなりうる.
\subsection{背景と課題}
\subsubsection{地域活性化における内発的発展}
近年,地域活性化は国土政策における重要な課題となっており,なかでも、条件不利地域の維持に関する調査は,論点や調査方法こそ異なるが,各省庁によって(国土交通省\cite{1},)実施されている.このような現状に対して,まち・ひと・しごと創生本部や内閣府地方創生推進事務局の設置などを皮切りに、国土交通省重点政策\cite{2}の策定を始め,地域おこし協力隊の派遣など国や自治体による政策・支援活動も数多く実施されている.しかし,トップダウンの財源と人的リソースは限られていることもあり,依然として担い手の確保の状況は厳しく,地域からの内発的な発展が強く求められている.
\subsubsection{過疎地域における「地域コミュニティ問題」の現状}
過疎化とは、簡単にいうと地域の人口 (戸数)が急減し、そのことで産業の衰 退や生活環境の悪化がもたらされ、住民意識が低下し、最後には地域から人がいなくなる(集落が消滅する)ことと捉えられる。\cite{3}これは、農林水産業等の第一次産業から第二・三次産業へ の移行とそれに伴う農村部から都市部への人口移動という中で、すでに1960年代の高度経済成長期から「問題」として捉えられてきた.\cite{4}さらに山下は、「過疎問題」という 問題提起そのものが政治・行政によってなされ、財政的支援による中央との格差是正として解決が求められていくことで、結果として、過疎地域自身が問題を内発的に解決せずに、 地域はつねに政策の客体=受け手として振る舞うよう習慣化されてきたことを指摘している.\cite{5}
 さらに過疎地域の地域コミュニティでは,集落組織の弱体化が進んでおり、機会の減少や閉鎖的な空間であること、変化がおこりにくいなどの過疎集落の特徴がある。これらの特徴から地域の現在と未来に対して悲観的になり、集落機能や集落そのものの維持に対する関心を失い、それが結果として実際に集落機能や地域ネットワーク、集落そのものの喪失を早めるという、「負のスパイラル」ともいうべき現象が起きていると言われている。\cite{6}ゆえに、集落の人々が主体的に取り組む気持ちの涵養が重要視され、集落に居住していない人々、その地の出身者や近くの集落、また中心都市の人々を、集落に積極的に関わらせていく必要があると山下は論じている。\cite{7}


\subsection{本研究の目的}
筆者はこれまでに述べた背景を踏まえ、過疎地域の内発的な発展を推進するために必要不可欠である地域再生にむけた地域住民のモチベーションを高めるため、「対話と交流の場」を過疎集落において形成する手法について考察する。そこで以下のように目的と達成要件を定義した.
\begin{itemize}
\item 研究目的 \\「対話と交流の場」形成によって過疎地域における地域コミュニティーを強化する.

\item 達成用件\\過疎地域において「対話と交流の”場”」形成する手法について考察し、交流の”場”におけるガイドラインの設定を行う.

\end{itemize}

\subsection{本論文の構成}
本論文の構成を以下に示す。
\begin{enumerate}
\item 序論 … 本研究の概要、背景と研究目的・達成用件について述べ、本論文の構成について説明する.
\item 先行研究と概念の整理 … 先行研究と概念の整理を行い、限界集落における地域活性化活動での本論文の位置付けを述べる.

\item 対象地域… 本研究で実践の場として横根地区を選定した理由を述べ,地域住民の地域関心を図るためにアンケート調査・現地調査を行い,それらの結果から対象地域について詳述する.
\item 交流の”場”におけるガイドラインの設定 … 先行研究に基づき、ウェブサイト構築ワークショップについて検討し,「過疎地域の“場”の形成(手法)における設計・運営の指針を定め,「パターン」にまとめる.

\item 交流の”場”におけるガイドラインの有効性の検証…新潟県魚沼市横根集落において、第4章で設定した指針をもとに“場”の実践を行い,有効性の検証を行う.
\item まとめ…本研究の結論および研究成果が持つ意義を述べる.新たな問題点や展望をまとめる.


\end{enumerate}

\newpage

\section{先行研究と概念の整理}
本章では、地域活性化の定義をした後、地域活性化活動と地域住民の主体性・愛着との関係性を既存の研究から明らかにする.さらに過疎地域における地域活性化活動での本論文の位置付けを述べる.
\subsection{地域活性化とは}
地域活性化という言葉の意味には、「担い手形成」と「生産動向」に着目し、地域農林業の活性化状況(農林業活性化度)を捉えた「農林業活性化」や各市町村の「人口動態」と「人口構成」に着目した概念など、定義は様々ある。過疎地域において考えた場合、集落が人口減少にあるなかで、産業を生むことや、交流人口が単純に増やすことを考えるには破綻がある。そこで本研究では、塩見による、「活性化とはそこに住む人びとが地域の資源を活用し、生きいきとした創造的な生活を営んでいる状態、またはそうした目標に向かって努力している状態を指すのであろう」という提唱を活性化の定義とする。
 また、このような地域活性化を促す活動は、地域への愛着が強い人ほど居住継続意識を示し地域活動へ積極的に参加する意識が高いことや、町内活動やまちづくり活動などの活動に熱心であること\cite{8}がわかっている。地域に対する地域活性化活動はその土地に対する愛着やシビックプライドの醸成シビックプライドが高いことという二点が挙げられており、これを高めるためには住民の主体性や地域当事者性 (オーナーシップ) を育んでいくことが重要となる.

\subsection{対話と交流の“場”}
“場”に関して, 清水博は, 一般に人々が身体を関与させながら共創的コミュニケーションをおこなう 「共創の舞台」 を ”場”と呼び\cite{9}, 伊丹敬之は, ”場”とは, 「人々がそこに参加し, 意識・無意識のうちに相互に観察し, コミュニケーションを行い, 相互に理解し, 相互に働きかけ合い, 相互に心理的刺激をする, その状況の枠組みのこと」 「人々の間の情報的相互作用と心理的相互作用の容れもの」 と定義している\cite{10}. さらに, 和田宰は, 「異なる価値観や能力を持つ『ひと』」 が, 相互作用を通じて創造的な活動を生み出していくため には, 『創発』 を生み出す相互作用の場をつくることが不可欠である」 「場が与えられることによって, それぞれの 『ひと』 は潜在的な価値観や能力を顕在化させ, 他の 『ひと』 との相互作用を通じて, 創造的な活動を生み出す可能性を得ることになる」 と”場”の重要性を指摘している\cite{11}.その上で、まちづくりの展開における”場”について, 久隆浩は, 地域に暮らす人々が集まり, 自由に意見交換や情報交換し, 楽しく気軽に話を展開し, その中から, 気づきが生まれ, 新たなつながりが生まれていく 「交流の場」 の重要性を指摘している\cite{12}これらの ”場”の定義からもわかるように,地域活性化を促す活動においては、単に参加できる”場”の形成ではなく,対話や交流を通じて相互作用や関係変容が起こる ”場”の形成が必要である. そうした”場”からつながりや輪が生まれ, 活動や事業 (アクションやプログラム), 組織が創発する. さらに, プロセスを通じてシビックプライド, 市民の主体性や地域当事者性の育みが進み, 新しい公共の創出につながっていくことが期待される.\\ “場”の形成手法については、ただ単純にひとが集まる”場”をつくれば, 自動的に相互作用や関係変容が起こり, 何かが生まれるというわけではなく、世田谷まちづくりセンターは, 参加のデザインとして, ①プロセスデザイン, ②プログラムデザイン, ③参加形態のデザインの三つのデザインを挙げている[12].さらに世古一穂は, ①参加のプロセスデザイン, ②参加のプログラムデザイン, ③参加構成のデザインの三つのデザインの重要性を挙げている.[13]状況に応じて, どのように場をデザインしていくのか, どのように運営していくのかという場づくりの方法が問われているといえる.




\subsection{参考文献の書き方}


\begin{thebibliography}{999}
  \bibitem{1}
  著者名1,著者名2
  論文タイトル名、
  発表シンポジウム・会議名、
  発表学会名、
  記事番号やページ、
  発表年
  \bibitem{2}
  著者名1,著者名2
  論文タイトル名、
  発表シンポジウム・会議名、
  発表学会名、
  記事番号やページ、
  発表年
  \bibitem{3}
  著者名1,著者名2
  論文タイトル名、
  発表シンポジウム・会議名、
  発表学会名、
  記事番号やページ、
  発表年
  \bibitem{4}
  著者名1,著者名2
  論文タイトル名、
  発表シンポジウム・会議名、
  発表学会名、
  記事番号やページ、
  発表年
  \bibitem{5}
  著者名1,著者名2
  論文タイトル名、
  発表シンポジウム・会議名、
  発表学会名、
  記事番号やページ、
  発表年
  \bibitem{6}
  著者名1,著者名2
  論文タイトル名、
  発表シンポジウム・会議名、
  発表学会名、
  記事番号やページ、
  発表年
  \bibitem{7}
  著者名1,著者名2
  論文タイトル名、
  発表シンポジウム・会議名、
  発表学会名、
  記事番号やページ、
  発表年
  \bibitem{8}
  著者名1,著者名2
  論文タイトル名、
  発表シンポジウム・会議名、
  発表学会名、
  記事番号やページ、
  発表年

\end{thebibliography}

\section*{論文の提出期限}
大学院は大学院教務委員に確認し、学部生は指導教員に確認してください。
\end{document}
